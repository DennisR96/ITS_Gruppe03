\documentclass[12pt]{article}

\usepackage{fancyhdr}                       % Package: Fancy HDR

\title{Projektdokumentation: Frequenz-Analyse und Hall}        % Titel
\author{David Yaman, Dennis Räk}            % Author
\date{IT-Systeme: SS 2021}                    % Datum/Kurs

\pagestyle{fancy}                           % eigener Seitenstil
\fancyhf{}                                  % alle Kopf- und Fußzeilenfelder bereinigen
\fancyhead[L]{ITS SS 2021}                  % Header links
\fancyhead[C]{}                             % Header Mitte
\fancyhead[R]{David Yaman, Dennis Räk}      % Header rechts
\renewcommand{\headrulewidth}{0.4pt}        % Trennlinie (Oben)
\fancyfoot[C]{\thepage}                     % Seitennummer
\renewcommand{\footrulewidth}{0.4pt}        % Trennlinie (Unten)


\begin{document}
\maketitle
\newpage
\section{Konzept}
\section{Funktionen}
\section{Umsetzung}
\section{Umsetzung im Programmcode}
\section{Probleme}
In diesem Abschnitt werden die Hindernisse und Probleme bei der Realisation des Projektes beschreiben.
\subsection{Audio-Board}
Anfänglich kam es zu Schwierigkeiten beim Anschluss des Audio-Boards. Über normale Jumper Wire ist eine störfreie Signalwiedergabe nicht möglich. 
Erst durch auflöten auf eine Lochrasterplattine ist eine Interferenzfreie Verbindung möglich und das Audio-Board funktioniert planmäßig.
\subsection{Einbinden von Effekten}
Im ursprünglichen Konzept sollte die eigenständige Programmierung von Audio Effekten erfolgen. 
Dieses Verfahren ist an mehreren Stellen gescheitert. 
\\
\\
Viele Effekte lassen sich leicht in höheren Programmiersprachen
wie Python oder MATLAB realisieren, aber für eine effiziente Programmierung in Echtzeit werden viele erweiterte C Kentnisse benötigt. 
Es ist zwar leicht das Konzept nachzubauen, aber ohne stabiles DSP Framework ist eine gute 
Performance nur schlecht möglicht. 
\\
\\
Der Teensy erweist sich bei diesem Verfahren ebenfalls als eine Herausforderung. Bei Überwindung dieser ersten Herausforderung
ist es    
\subsection{Verlust eines Teensy Boards}
Beim Anschluss der Drehencoder ist eine fehlerhafte Anbindung an . 
Da die Teensy PINs nicht 5V tolerant sind, führte dies zu einer Zerstörung des Teensy Boards.
Es musste ein Ersatzgerät erworben werden. 

  
\section{Fazit}
Alles in allem ist das Projekt erfolgreich gewesen. 

\end{document}